\section{2018-02-09}

\subsection{Functors and Function classes}

The \cpp{()} operator - use as a ``functor'' or ``function class''. Effectively, you wrap up a function in a class.

Example:

\begin{lstlisting}[language=C++]
class bell_curve {
  private:
    float m_ht;
    float m_wd;

  public:
    bell_curve(float ht = 1, float wd = 1) : m_ht(ht), m_wd(wd) { }
    float operator()(float x) { return m_ht / (m_wd+x*x); }
};

//...

bell_curve y(1, 1);
//...
cout << y(2);
\end{lstlisting}


\subsection{Friend functions}

A function is a friend of a class that it has been given direct access rights to the private section of that class. Words of wisdom: \textbf{Don't trust your friends... Avoid friends.}

Note: The only place that \cpp{friend} should ever appear is \underline{inside} a class definition.

A function that (almost) everyone makes a non-member friend is the streaming operators.

``Stoopit'' code example:

\begin{lstlisting}[language=C++]
class A {
  public:
    ostream& operator<<(ostream& os) const {
      //...
    }
};

A a;
a << cout;  //gross code
\end{lstlisting}


\subsection{Static functions}

Add functionality for classes that doesn't need (or can't have) a calling object

\begin{lstlisting}[language=C++]
//complex.h
class complex {
  public:
    complex(float re, float im) : m_re(re), m_im(im) { }
    static complex fileread(istream& in = cin);
}; 

//complex.cpp
complex complex::fileread(istream& in) {
  float r, i;
  in >> r;
  in >> i;
  return complex(r, i);
}

//call
complex z = complex::fileread(in);
\end{lstlisting}

Make data types that simply wrap around static functions:

\begin{lstlisting}[language=C++]
//definition
struct stooge {
  static void moe();
  static void larry();
  static void curly();
};

//call
stooge::curly();
\end{lstlisting}


\subsection{Conversions}

If your class has a constructor that takes a single variable of some type and builds an object (of course), than it can be thought of as a ``constructor for automatic conversion''. 

\begin{lstlisting}[language=C++]
complex operator+(const complex& lhs, const complex& rhs) {
  z = u + r; // \
  z = u + 4; // | All work
  z = r + u; // /
}
\end{lstlisting}

Functions for automatic conversion are called ``conversion functions''.

\begin{lstlisting}[language=C++]
class complex {
  private:
    float m_re, m_im;
  
  public:
    //...
    operator bool() { return m_re || m_im; }
}

float x;
complex z;
//...
if (z) {
  //do stuff
}
\end{lstlisting}
