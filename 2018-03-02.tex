\section{2018-03-02}

\subsection{Vector and matrix norms}

\subsubsection{Definition of a norm}

A \textbf{norm} is a way to measure. But we demand certain properties.

\underline{Definition:} Suppose $V$ is a vector space. Then $N: V \rightarrow \mathbb{R}$ is a norm if:

\begin{enumerate}
  \item $N(\bar{v}) \geq 0; \forall \bar{v} \in V; N(\bar{v}) = 0 \text{ iff } \bar{v} \equiv 0$
  \item $N(\alpha \bar{v}) = |\alpha| N(\bar{v}); \forall \bar{v} \in V \text{ and } \alpha \in \mathbb{R}$
  \item $N(\bar{v}+\bar{w}) \leq N(\bar{v}) + N(\bar{w}); \forall \bar{v}, \bar{w} \in V$
\end{enumerate}

Example - \textbf{p-norms}:

\begin{align*}
                  &||\bar{v}||_p      = (\sum_{i=1}^{n} |x_i|^p)^{1/p} \\
  p = 1:\ \       &||\bar{v}||_1      = (\sum_{i=1}^{n} |x_i|^p) \\
  p = 2:\ \       &||\bar{v}||_2      = \sqrt{\sum_{i=1}^{n} |x_i|^p} \\
  p = \infty:\ \  &||\bar{v}||_\infty = \text{max}_i |x_i|
\end{align*}

\subsubsection{Theorem: Equivalence of norms}

Let $N$ and $M$ be norms on $\mathbb{R}^n$. Then there exist constants $c_1 > 0$ and $c_2 > 0$ such that:

\begin{align*}
  c_1 M(\bar{x}) \leq N(\bar{x}) \leq c_2 M(\bar{x}); \forall \bar{x} \in V = \mathbb{R}^n
\end{align*}

Lemma (continuity of norms):

Let $M(\bar{x})$ be a norm on $\mathbb{R}^n$ Then $M(\bar{x})$ is a continuous function of the $x_i; \forall \bar{x} \in \mathbb{R}^n (\bar{x} = <x_1, x_2, ..., x_n>)$

\subsubsection{A proof}

Show that if $x_i \approx y_i$, then $N(\bar{x}) \approx N(\bar{y})$

We have: $(\bar{x} - \bar{y}) = \sum_{i=1}^{n} (x_i - y_i) e_i$

\begin{align*}
  N(\bar{x} - \bar{y}) = ||\bar{x} - \bar{y}|| \leq \sum_{i=1}^{n} |x_i - y_i| N(e_i) \leq \sum_{i=1}^{n} \text{max}_i |x_i - y_i| N(e_i)
\end{align*}

Now, using The Reverse Triangle Equality:

\begin{align*}
  |N(\bar{x} - N(\bar{y}))| \leq N(\bar{x} - \bar{y}) \leq ||\bar{x} - \bar{y}||_\infty c; c = \sum_i N(e_i)
\end{align*}

Or:

\begin{align*}
  |N(\bar{x} - N(\bar{y}))| \leq c ||\bar{x} - \bar{y}||_\infty
\end{align*}

\subsubsection{Proof of equality of norms}

It is sufficient to show for an arbitrary norm $N$ and a special norm, $||\dot||_infty$. Thus, show $\exists c_1, c_2 > 0$ such that:

\begin{align*}
  c_1 ||\bar{x}||_\infty \leq N(\bar{x}) \leq c_2 ||\bar{x}||_\infty
\end{align*}

From the lemma, if we set $\bar{y} = 0$ we have:

\begin{align*}
  |N(\bar{x}) - N(\bar{y})|  & \leq c_2 ||\bar{x} - \bar{y}||_\infty \\
  N(\bar{x}) = |N(\bar{x})| & \leq c_2 ||\bar{x}||_\infty \\
  N(\bar{x})                & \leq c_2 ||\bar{x}||_\infty
\end{align*}

So, the upper inequality holds. The lower inequality holds because all asdf are asdf away from 0. <Unable to read what Price wrote>

\begin{align*}
  c_1 ||\bar{x}||_\infty &\leq N(\bar{x}) \leq c_2 ||\bar{x}||_\infty \\
  c_3 ||\bar{x}||_\infty &\leq M(\bar{x}) \leq c_4 ||\bar{x}||_\infty
\end{align*}

So what?! What does this theorem give us? When trying to determine something like:

\begin{align*}
  \lim_{i \rightarrow \infty} \bar{v}_i \rightarrow \bar{v}\ \Leftrightarrow\ |\bar{v}_i - \bar{v}| \rightarrow 0
\end{align*}

It doesn't matter what norm you use! Any one can be used to prove convergence.