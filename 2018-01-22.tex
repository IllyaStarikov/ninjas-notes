\section{2018-01-22}


\subsection{Macros}

I sort of think of macros like I do constants.

\begin{enumerate}
  \item Use the \shellcmd{=} to build a macro (ex. \shellcmd{CXX = /mst/bin/g++}).
  \item Don't use tabs or \shellcmd{:}.
  \item The convention: use uper case to name a macro.
  \item To evaluate a macro, use withen \shellcmd{\$\{----\}} or \shellcmd{\$(----)}.
  \item For a single-character macro, the \shellcmd{\{\}} or \shellcmd{()} are not necessary.
  \item \shellcmd{BOB =} would assign the empty string to BOB
  \item Order in definition doesn't matter. However, if your macro defns recursively, then \shellcmd{make} will get "upset".
\end{enumerate}

\subsection{Internal macros}

\begin{itemize}
  \item \shellcmd{\$\{cc\}} - evaluates to C compiler
  \item \shellcmd{\$\{cxx\}} - evaluates to C++ compiler
  \item \shellcmd{\$@} - current target
  \item \shellcmd{\$?} - list of prerequisites newer than the current target
  \item \shellcmd{\$<} - standard input
\end{itemize}

Macros can be used on the command line (ex. \shellcmd{make FLAGS=hello})

\subsection{The priority of macro assignments}
 
(Order is least to greatest)

\begin{enumerate}
  \item The default defns of make
  \item current shell environment variables
  \item description file defns
  \item command line defns
\end{enumerate}

Note: \shellcmd{make -e} will force \#2 to be dominant except for command line

\subsection{Macro string substitutions}

If \shellcmd{SOURCES = moe.cpp larry.cpp} then \shellcmd{ls \$\{SOURCES : .cpp = .o\}} -> \shellcmd{moe.o larry.o}

\subsection{Suffix rules}

A \textbf{suffix rule} is a predefined generalized makefile entry.

\begin{lstlisting}[language=make]
SUFFIXES : .cpp
.cpp .o : $<
  ${cxx} -c ${FLAGS} $< -o $@
\end{lstlisting}

How does make work? In Unix, we have conventions. A .c extension signals a C-compilable file. Conventions and these form the basis of rules on how to build files. 

So, suffix rules are either built-in rules or user-defined rules on how to "do something".

Backward chaining is employed to build any prereq of the current target(s). And the re-build is determined by user-defined and automatic/internal suffix rules.

\begin{itemize}
  \item First looks for specific user-defined rule.
  \item Resort to internal rules.
\end{itemize}

Then, \shellcmd{make} looks for any file in the current directory that:

\begin{itemize}
  \item has the same root name
  \item has a signifcant suffix
  \item is required to build the target
\end{itemize}