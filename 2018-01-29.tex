\section{2018-01-29}

\subsection{The Array class}

\begin{enumerate}
  \item runtime sizing - using the constructs
  \item use of the `[]` operator for indexing into the array
  \item automatic memory management
  \item array copying facility
  \item being able to assign a single value to ever element of the array
  \item resizing of my array
\end{enumerate}

\subsubsection{array.h}

\begin{lstlisting}[language=C++]
template <typename T>
class Array {
  public:
    Array(const int size);
    Array();
    Array(const Array& source);
    ~Array();
    T& operator[](const int i);
    int getSize() const;
    Array& operator=(const Array& source);
    Array& operator=(const T s);
    void setSize(const int size);

  private:
    int m_size;
    T* ptr_to_data;
    void arraycopy(const Array& source);
};
\end{lstlisting}

\subsubsection{array.cpp}

\begin{lstlisting}[language=C++]
Array::Array(const int size) {
  m_size = size;
  ptr_to_data = new T[size];
}

Array::Array(const Array& source) {
  m_size = source.m_size;
  ptr_to_data = new T[m_size];
  arraycopy(source);
}

void Array::arraycopy(const Array& source) {
  T* p = ptr_to_data + m_size;
  T* q = source.ptr_to_data + m_size;
  while (p > ptr_to_data)
    *--p = *--q;
  return;
}

Array::~Array() { delete[] ptr_to_data; }

float& Array::operator[](const int i) {
  return ptr_to_data[i];
}

Array& Array::operator=(const Array& s) {
  if (ptr_to_data != s.ptr_to_data) {
    setSize(s.m_size);
    arraycopy(s);
  }

  return *this;
}

void Array::setSize(const int size) {
  if (size != m_size) {
    delete[] ptr_to_data;
    m_size = size;
    ptr_to_data = new float[size];
  }
}
\end{lstlisting}