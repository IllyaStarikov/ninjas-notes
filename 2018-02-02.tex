\section{2018-02-02}

\subsection{Continuing review}

\subsubsection{Exception handling}

What happens when an object is thrown? Well, two things are thrown:

\begin{itemize}
  \item an object
  \item control
\end{itemize}

An example on how to properly make and use exceptions:

\begin{lstlisting}[language=C++]
class RangeErr {
  public:
    RangeErr(int i);
    int BadSubscript();
  private:
    int subscr; 
};

class SizeErr {
  public:
    SizeErr(int i);
    int BadSubscript();
  private:
    int subscr; 
};
\end{lstlisting}

We would want to modify 3 facts int he Array class. First:

\begin{lstlisting}[language=C++]
template <class T>
Array<T>::Array(int n) {
  if (n < 0) throw SizeErr(n);
  m_size  = n;
  ptr_to_data = new T[n];
}

template <class T>
T& Array<T>::operator[](int i) {
  if (i < 0 || i >= m_size) throw RangeErr(i);
  return ptr_to_data[i];
}

template <class T>
void Array<T>::setSize(int n) {
  if (n != m_size) {
    if (n < 0) throw SizeErr(n);
    //...
  }
}
\end{lstlisting}

Later...

\begin{lstlisting}[language=C++]
try {
  int n;
  cin >> n;
  Array<float> array(n);
  //...
  int j;
  cin >> j;
  float f = array[j];
  //...
}
catch (RangeErr e) {
  cerr << "Subscript out of range: " << e.BadSubscript();
}
catch (SizeErr e) {
  cerr << "Size of declared array is negative: " << e.BadSubscript();
}
\end{lstlisting}

\subsection{Math stuff}

\subsubsection{Functions}

\begin{equation*}
\end{equation*}