\section{2018-02-05}

\subsection{More stuff}

\subsubsection{Reference returns}

\begin{lstlisting}[language=C++]
int index_of_min(const Array<float>& a) {
  //...
  return index_of_min_elem;
}

float& min_element(Array<float>& a) {
  return a[index_of_min];
}
\end{lstlisting}

This return by reference, though legal andpreferred in some cases, can lead to heartbreak. This is because that reference creates a ``backdoor'' into the class.

\begin{lstlisting}[language=C++]
Array<float> a(10);
//...
min_element(a) = 0; //changing the min. elem
\end{lstlisting}

Consider not returning a reference. When you returna non-reference value, you're actually returning a copy - hence, a constant. Alternatively, write a second version of the function that has a const parameter and a const reference return.

Also: Be careful NOT to return a reference to a local object/value.


\subsubsection{Function overloading vs./and templates}

You overload a function when you want to have multiple functions with the same name but different parameter lists and so different functionalities for different types.

\begin{enumerate}
  \item Overloading functions and how C++ resolves functions calls.
There are levels of consideration that C++ will go through to know how to resolve a function call.
  \begin{enumerate}
    \item Consider an exact match with only the possibility of trivial conversions (e.g. \cpp{T[]} to/from \cpp{T*}, \cpp{T} and \cpp{T\&})

    This implies that \cpp{void f(T\&);} and \cpp{void f(T);} \underline{cannot} co-exist in the same program.
    
    Note: \cpp{void f(const T);} and \cpp{void f(T);} \underline{can} co-exist.

    \item Match w/ promotions

    char, enum, short <-> int, long; float <-> double (e.g. \cpp{void bob(float);} and \cpp{void bob(int);} can co-exist)

    \item Standard conversions: int-types to float-types

    \item User-defined conversions. Be careful that your code only defines one way to convert from one type to another. (e.g. \cpp{void bob(int, float);} and \cpp{void bob(int, double);})

    What does \cpp{bob(1, 2);}? C++ will flag an error - ambigious funtion call

    \textbf{If you need to search the rules for function call resolution you need to rewrite your code.}
  \end{enumerate}

  \item Templates

  Suppose we have:

\begin{lstlisting}[language=C++]
template <class T>
void bob(T a, const T& b);
\end{lstlisting}

  Member of the class include
  
\begin{lstlisting}[language=C++]
void bob(int a, const int& b);
void bob(char a, const char& b);
\end{lstlisting}

\end{enumerate}

Now the steps c++ goes through to resolve function calls:

\begin{enumerate}
  \item Match without the possibility of any conversion.
  \item Template match with possibility of trivial conversion.
  \item Function declaration using ordinary matching processes.
\end{enumerate}

Example:

\begin{lstlisting}[language=C++]
template <class T>
void bob(T a, const T& b);

int a;
int b;

bob(1,   a); //matches the template w/ T=int
bob(1.0, b); //matches the template w/ T=float
bob(1,   b); //no match
\end{lstlisting}

Now add in:

\begin{lstlisting}[language=C++]
void bob(float a, const float& b);
\end{lstlisting}

Now, \cpp{bob(1, b);} matches with promotion.

Now for specialization:

\begin{lstlisting}[language=C++]
template <class T>
int compare(const T& t1, const T& t2) {
  if (t1 < t2) return -1;
  if (t2 < t1) return 1;
  return 0;
}

template <> //specialization
int compare<const char*(const char* const& s1, const conar* const& s2) {
  return strcmp(s1, s2);
}
\end{lstlisting}

\subsection{Member functions}

Distinguish 'twixt and between non-member and member functions.
One difference is that member functions can be const functions. A const function is a member function that can NOT alter the calling object. If you want to overload the \cpp{[]} operator, make 2 revisions: a const version and a non-const version.

Some stuff to know:

\begin{itemize}
  \item no return type 
  \item no return statement
  \item named name of class
  \item C++ with provide a default constructor but that is suppressed when you create ANY constructor.
\end{itemize}

\begin{itemize}
  \item 
\end{itemize}