\section{2018-02-28}

\subsection{Modified Gram-Schmidt Method}

(Copied from handout)

From the original set $T = {v_1, v_2, ..., v_n}$ we produce the final orthonormal set $S = {u_1, u_2, ..., u_n}$. But along the way we also produce intermediate sets of $W^j = {u^j_j+1, ..., u^j_n}$. These are created by adjusting the remaining vectors in T using the newest addition to $S$.

To start, let $u_1 = v_1$ and then $u_1 = \frac{u_1}{\sqrt{(u_1, u_1)}}$. Continue by adjusting the remaining vectors in $T$: $u^1_j = v_j - (v_j, u_1)u_1$ for $j = 2, 3, ... n$.

Now, $u_2 = \frac{u^1_2}{\sqrt{(u^1_2, u^1_2)}}$ and $u^2_j = u^1_j - (u^1_j, u_2)u_2$ for $j = 3, 4, ..., n$

Continue in this manner, we produce the orthonormal set we desired in the first place, but without the out-of-control loss of significance.

In general, once we have ${u_1, ..., u_{k-1}}$, we set $u_k = \frac{u^{k-1}_k}{\sqrt{(u^{k-1}_k, u^{k-1}_k)}}$ and $u^k_j = u^{k-1}_j - (u^{k-1}_j, u_k)u_k$, for $j = k+1, ..., n$.