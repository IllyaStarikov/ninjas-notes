\section{2018-04-06}

\subsection{Member function forwarding}

\subsubsection{More Acme130}

\begin{lstlisting}[language=C++]
class Acme130 : public VoltageSupply, public GPIBInstrument {
  public:
    Acme130(GPIBController& c, int address);

    //VS interface
    //...

    //GPIBInstrument interface
    virtual void send(/* ... */);
    virtual void send (/* ... */);
    virtual float receive();

  private:
    GPIBController& myController;
    int myAddress; 
};
\end{lstlisting}

Well, the \cpp{Volton59} looks just like... also derives from GPIBInstrument. Likewise, the VoltyMetric class derives from \cpp{GPIBInstrument}. All these derive from \cpp{GPIBInstrument} and use the same functions and data. We can avoid all this replication by placing it all in another class and employing \underline{function forwarding}. 


\begin{lstlisting}[language=C++]
class GPIBInstrumentData {
  public:
    GPIBInstrumentData(GPIBController& c, int address, const char* name);

    //Instrument interface
    void send(const char*);
    void send(float f);
    float receive();

  private:
    GPIBController& myController;
    int myAddress;
};
\end{lstlisting}

Note: It is structurally the same as the \cpp{Acme130}, \cpp{Volton59}, and \cpp{VoltyMetric}, but not like the interface bases. The function implementations are the same as was in the \cpp{Acme130}, \cpp{Volton59}, and \cpp{VoltyMetric}, but here we now have just one version. 

\begin{lstlisting}[language=C++]
void GPIBInstrumentData::send(float val) {
  myController.send(myAddress, val);
}
\end{lstlisting}

The constructor is going to initialize the data and it inserts the device onto the controller. But, since it isn't for a particular device, it has to have a third parameter and that's the name of the device.

\begin{lstlisting}[language=C++]
GPIBInstrumentData::GPIBInstrumentData(GPIBController& c, int address, const char* name) 
  : myController(c), myAddress(address) {
  myController.insert(name, address);
}

class Acme130 : public VoltageSupply, public GPIBInstrument {
  private:
    GPIBInstrumentData gpibRep;

  public:
    Acme130(GPIBController& controller, int address) : gpibRep(controller, address, "Acme130") { }

    //VS interface
    //...

    //Instrument interface
    virtual void send(/* ... */) { gpibRep.send(/* ... */); }   
    virtual void send(/* ... */) { gpibRep.send(/* ... */); }
    virtual float receive(/* ... */) { return gpibRep.receive(); }
};

//repeated for Volton59 and VoltyMetric
\end{lstlisting}

\subsubsection{A new UML diagram}

\begin{tikzpicture}
  \begin{class}{GPIBInstrument}{6, 8}
  \end{class}

  \begin{class}{VoltyMetrics}{12, 4}
    \inherit{GPIBInstrument}
  \end{class}
  
  \begin{class}{Volton59}{6, 4}
    \inherit{GPIBInstrument}
  \end{class}
  
  \begin{class}{Acme130}{0, 4}
    \inherit{GPIBInstrument}
  \end{class}

  \begin{class}{GPIBInstrumentData}{6, 0}
  \end{class}

  \unidirectionalAssociation{Acme130}{}{}{GPIBInstrumentData}
  \unidirectionalAssociation{Volton59}{}{}{GPIBInstrumentData}
  \unidirectionalAssociation{VoltyMetrics}{}{}{GPIBInstrumentData}
\end{tikzpicture}