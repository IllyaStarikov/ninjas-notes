\section{2018-02-21}

\subsection{Smart Pointers}

We all know that pointers are necessary to programming in C++. But the built-in pointers are too weak and too strong: they allow the programmer too much power with too little oversight.

\textbf{Aggregation} is the combining of different objets into another object. And when you ``contain'' a pointer, this is called \textbf{referential aggregation}. The problem is that the life of the pointer and what it points to don't necessary coincide. So, when objects get copied (especially by built-in copy constructors), then the possibility of a shallow is there. (This is called ``referential aggregation''). Leaks can be caused in many ways... including ways you hadn't thought of... yet. And they can happen when you think you have everything covered, like a user-defined copy constructor and assign operator.

Programmer-defined objects that wrap up pointers are called \textbf{smart pointers} are designed to do what they're supposed to do and clean up after themselves. We write these smart pointers to preserve a 1-1 relationship between the pointer and what it points to.

\begin{lstlisting}[language=C++]
template <typename T>
class CBPtr { //copied builtin pointer
  protected:
    T* theP;

  public:
    CBPtr() : theP(0) { }
    CBPtr(T* justNewed) : theP(justNewed) { }
    ~CBPtr() { delete theP; }
    CBPtr(const CBPtr<T>& aCP) : theP(aCP.isNull() ? 0 : new T(*aCP.theP)) { }

    CBPtr<T>& operator=(T* p) {
      delete theP;
      theP = p;
      return *this;
    }

    CBPtr<T>& operator=(const CBPtr<T>& rhs) {
      if (theP != rhs.theP) {
        delete theP;
        theP = rhs.isNull() ? 0 : new T(*rhs.theP);
        return *this;
      }
    }

    T& operator*() const { return *theP; }

    bool isNull() const { return theP == 0; }

    T* releaseControl() {
      T* saveP = theP;
      theP = 0;
      return saveP;
    }

    friend bool operator==(const CBPtr<T>& lhs, const CBPtr<T>& rhs) {
      return lhs.theP == rhs.theP;
    }

    friend bool operator!=(/* etc... */) {
      //...
    }
};
\end{lstlisting}

Benefits:

\begin{itemize}
  \item one-to-one relationship between pointer and heap object
  \item automatic copy is made when C++ auto-generated copy constructor is called since the copy constructor for \cpp{CBPtr} is called
  \item deleting the pointer deletes the heap object \textrightarrow no garbage
  \item overloads to \cpp{*} operator
\end{itemize}

\begin{lstlisting}[language=C++]
template <class T>
class COPtr : public CBPtr<T> {
  public:
    COPtr(const COPtr<T>& p) : CBPtr<T>(p) { }
    COPtr() : CBPtr<T>() { }
    COPtr(T* justNewed) : CBPtr<T>(justNewed) { }
    
    COPtr<T>& operator=(T* rhs) {
      CBPtr<T>::operator=(rhs);
      return *this;
    }

    COPtr<T>& operator=(const COPtr<T>& rhs) {
      CBPtr<T>::operator=(rhs);
      return *this;
    }

    T* operator->() const { return theP; }
}
\end{lstlisting}

Note: the \cpp{CBPtr} class is for built-in types. It doesn't add (and should not) have an overload of the \cpp{->} operator. This operator should return either a built-in pointer to a class object or an object of a pointer-like class with an overload of the \cpp{->} operator. (Built-in pointers to built-in objects cannot be returned.) An example:

\begin{lstlisting}[language=C++]
class Leaker {
  public:
    Leaker(int size) : p(new float(0), array(size) { }
    Leaker(const Leaker& lkr) : p(new float*(lkr.p)) { }
    ~Leaker() { delete p; }
    
    Leakaer& opeartor=(const Leaker& lkr) {
      p     = new float(*lkr.p);
      array = lkr.array;
      
      return *this;
    }

  private:
    float * p;
    Array<int> array; 
};

//...

void f() {
  Leaker l(-1);
  //...
}
\end{lstlisting}

When \cpp{f} is called, the constructor for Leaker is called. The float is allocated and we try to construct the array with a negative size. An exception is thrown! Since all \textbf{fully constructed} objects get deallocated here, \underline{we have a leak}.

\begin{lstlisting}[language=C++]
class NonLeaker {
  public:
    NonLeaker(int size) : p(new float(0)), array(size) { }
    //...

  private:
    CBPtr<float> p;
    Array<int> array;
};
\end{lstlisting}