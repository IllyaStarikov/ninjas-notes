\section{2018-04-27}

\subsection{Traits classes}

\subsubsection{A short motivation}

With traits classes you can create a mapping of type names that can be very useful. Some examples:


\subsubsection{Example 1}

Suppose that you are writing pieces of a code library that is templatable on numeric types such as floats, doubles, long doubles, etc. Each type has a set of constants associated with it: a max exponent value, an ``epsilon'', a max mantissa size, etc. - all defined in some header file (somewhere). The confusion occurs when, often templating on a specific type, you need \cpp{FLT_MAX_EXP}, \cpp{DBL_MAX_EXP}, etc. The compiler needs a way to determine which you want. Here's the traits solution to the problem:

\begin{lstlisting}[language=C++]
template <class numT>
struct float_traits { };

struct float_traits<float> {
  typedef float float_type;
  enum {max_exponent = FLT_MAX_EXP};
  static inline float_type epsilon() { return FLT_EPSILON; }
  //...
};

struct float_traits<double> {
  typedef double float_type;
  enum {max_exponent = DBL_MAX_EXP};
  static inline float_type epsilon() { return DBL_EPSILON; }
}
\end{lstlisting}

...and etc. So, when referring to a \cpp{max_exponent} without knowing the type, we have no worries. For example:

\begin{lstlisting}[language=C++]
template <class numT>
class matrix {
  public:
    typedef numT num_Type;
    typedef float_traits<num_Type> traits_type;

    inline num_Type epsilon() { return traits_type::epsilon(); }
};
\end{lstlisting}


\subsubsection{Example 2}

The \cpp{average()} function:

\begin{lstlisting}[language=C++]
template <class T>
T average(const T* data, const int n) {
  T sum = 0;
  for (int i = 0; i < n; i++) sum += data[i];
  return sum/n;
}
\end{lstlisting}

This works just fine if you want the return type to be the same as the template. But what if you don't?

\begin{lstlisting}[language=C++]
template <class T>
struct float_trait { typedef T T_float; }

template <>
struct float_trait<char> {
  typedef double T_float;
};

template <>
struct float_trait<int> {
  typedef double T_float;
};

//etc.
\end{lstlisting}

So, what you map \underline{from} is the template parameter, and \underline{to} is what you put inside the struct. So then, \cpp{float_trait<T>::T_float} will evaluate to the appropriate float type for that \cpp{T_type}. So, the \cpp{average} function becomes:

\begin{lstlisting}[language=C++]
template <class T>
typename float_trait<T>::T_float average(const T* data, const int n) {
  typename float_trait<T>::T_float sum = 0;
  for (int i = 0; i < n; i++) sum += data[i];
  return sum/n;
}
\end{lstlisting}


\subsubsection{Example 3 (type promotion)}

\begin{lstlisting}[language=C++]
template <class T1, class T2>
Vector<???> operator+(Vector<T1>& a, Vector<T2>& b);
\end{lstlisting}

What will be the type for the return? The solution:

\begin{lstlisting}[language=C++]
template <class T1, class T2>
struct promote_trait { };

#define DECLARE_PROMOTE(A, B, C) \
  template<> struct promote_trait<A, B> \
  { typedef c T_promote; };

DECLARE_PROMOTE(int, char, int);
DECLARE_PROMOTE(int, flaot, float);
//etc.
\end{lstlisting}

The use:

\begin{lstlisting}[language=C++]
template <class T1, class T2>
Vector<typename promote_trait<T1, T2>::T_promote> operator+(Vector<T1>& a, Vector<T2> b);
\end{lstlisting}
