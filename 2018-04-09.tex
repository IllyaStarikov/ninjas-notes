\section{2018-04-09}

\subsection{Implementation using inheritance}

\subsubsection{Reworking GPIBInstrumentData}

We now use inheritance/derivation to accomplish what we set out to do with function forwarding.First, change the \cpp{GPIBInstrumentData} to be derived from \cpp{GPIBInstrument} so that it implements our interface.

\begin{lstlisting}[language=C++]
class GPIBInstrumentData : public virtual GPIBInstrument {
  public:
    GPIBInstrumentData(GPIBController& c, int address, const char* name);
    
    //GPIB interface
    virtual void send(/* ... */);
    virtual void send(/* ... */);
    virtual float receive();

  private:
    GPIBController& myController;
    int myAddress;
};
\end{lstlisting}

So, the \cpp{GPIBInstrumentData} ``is usable as'' a \cpp{GPIBInstrument}. The \cpp{GPIBInstrument} is a ``virtual base''. The implementations of the functions remain the same. Now, we rewrite the \cpp{Acme130}.

\begin{lstlisting}[language=C++]
class Acme130 : public virtual VoltageSupply, public virtual GPIBInstrument, private GPIBInstrumentData {
  public:
    Acme130(/* .. */);

    //VoltageSupply interface
    virtual void set(/* ... */);
    virtual float min();
    virtual float max();
};
\end{lstlisting}

Note:

\begin{itemize}
  \item The virtual derivations from base classes
  \item The private derivation from the implementation base
  \item Any instance of an \cpp{Acme130} contains a base subobject of the base
  \item Because of the private derivation, it is NOT ``usable as'' a \cpp{GPIBInstrumentData} object
  \item The private derivation ``hides'' the implementation of this class
\end{itemize}

Notice, the \cpp{Acme130} does \underline{not} implement the instrument interface functions. THose are privately inherited from the \cpp{GPIBInstrumentData}. Public access of the member functions is blocked. Clients have to access the functionality through that interface. They have to access through the \underline{same} interface base as the \cpp{GPIBInstrumentData}. And this is the reason for the virtual derivation. Function calls are routed through the implementation base.

\begin{tikzpicture}
  \begin{class}{GPIBInstrument}{8, 4}
  \end{class}

  \begin{class}{VoltageSupply}{0, 4}
  \end{class}
  
  \begin{class}{Acme130}{0, 0}
  \end{class}
  
  \begin{class}{GPIBInstrumentData}{8, 2}
  \end{class}

  % why does this package suck so hard?
  \draw [umlcd style dashed line, -|>] (Acme130) --node[above, sloped, black]{} (GPIBInstrument);
  \draw [umlcd style dashed line, -|>] (Acme130) --node[above, sloped, black]{} (VoltageSupply);
  \draw [umlcd style dashed line, -|>] (Acme130) --node[above, sloped, black]{private} (GPIBInstrumentData);
  \draw [umlcd style dashed line, -|>] (GPIBInstrumentData) --node[above, sloped, black]{} (GPIBInstrument);
\end{tikzpicture}


\subsubsection{}

Here's some example code that doesn't work:

\begin{lstlisting}[language=C++]
Acme130 s(gpib, 4);
s.send(10);
\end{lstlisting}

We can't do this. The private derivation prohibits it. What we can do is:

\begin{lstlisting}[language=C++]
GPIBInstrument* g = new Acme130(gpib, 4);
g->send(10);
//or
GPIBInstrument& t = *g;
t.send(10);
\end{lstlisting}


The whole idea is that you use the functionality of the particular deices through the interface. You can also ``override'' the privateness inherited from that implementation base by declaring something public:

\begin{lstlisting}[language=C++]
class Acme130 : public virtual VoltageSupply, public virtual GPIBInstrument,
                private GPIBInstrumentData {
  public:
    GPIBInstrumentData::send; //publicly expose GPIBID's inherited send function
    //...
};
\end{lstlisting}

Thus, all the code in the example will work (even the one that didn't before!).