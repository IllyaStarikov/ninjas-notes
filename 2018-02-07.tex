\section{2018-02-07}

\subsection{Constructors}

You initialize if:

\begin{enumerate}
  \item if you have no default constructor
  \item if you have a member variable that is a constant
  \item if you have a member variable that is a reference
\end{enumerate}

You don't have to initialize if...

\begin{enumerate}
  \item members are C++ primitives
  \item have a user defined default constructor
  \item have no user defined other constructor 
\end{enumerate}

\subsection{Copy constructors}

\cpp{A::A(const A& s);}

Copy constrcutors are called...

\begin{itemize}
  \item Explicitly
  \item pass-by-value parameters in functions
  \item return-by-value from a function
  \item exception is thrown
\end{itemize}

C++ provides an automatic copy constructor But, it will copy member-by-membr even if they are pointers. This leads to ``shallow'' copies - copying the pointer and not what it points to.

Remember: if your class contains a pointer (referential assumption) then you should write your own...

\begin{itemize}
  \item copy constructor
  \item copy assignment
  \item destructor
  \item perhaps, a defeault constructor
\end{itemize}

\subsection{Operators}

...aka function for some special symbol

\underline{Overloading} operators? There are some limitations:

\begin{itemize}
  \item Can't create new operators (pg. 31)
  \item Can't change the functionality of an operator for primitives.
  \item Can't change the arity of an operator
  \item Can't change the order of precedence
  \item Can't overload \cpp{::}, \cpp{.}, \cpp{.*}, \cpp{sizeof}, \cpp{?:}
\end{itemize}

Ground rules:

\begin{itemize}
  \item Make your operators make sense
  \item Define pairs when appopriate 
\end{itemize}

Example:

\begin{lstlisting}[language=C++]
complex& complex::operator+=(const complex& rhs) {
  m_re += rhs.m_re;
  m_im += rhs.m_im;
  return *this;
}

complex complex::operator+(const complex& rhs) const {
  complex result(lhs);
  return result += rhs;
}
\end{lstlisting}

\cpp{[]}, \cpp{()}, and \cpp{=} are special in that any overloads of them must be member functions of the class involving them.