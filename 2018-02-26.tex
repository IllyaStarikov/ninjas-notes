\section{2018-02-26}

\subsection{Inner Product Spaces}

\subsubsection{Some definitions}

\underline{Definition:} An \textbf{inner product} on a vector space $V$ is a real-valued function over $V$ with

\begin{enumerate}
  \item $(\alpha, \alpha) > 0;\ \forall \alpha \in V$ 
  \item $(\alpha, \beta) = (\beta, \alpha);\ \forall \alpha, \beta \in V$ 
  \item $(\alpha + \beta, \gamma) = (\alpha, \gamma) + (\beta, \gamma)$
  \item $(c\alpha, \beta) = c(\alpha, \beta);$ c a scalar
\end{enumerate}

\subsubsection{Theorem: Cauchy-Schwarz Inequality}

For any $\alpha, \beta \in V, (\alpha, \beta)^2 \leq |\alpha|^2 |\beta|^2$, where $|\alpha| = \sqrt{(\alpha, \beta)}$

\underline{Proof:}

If $\alpha = \emptyset$, then $(\alpha, \beta) = 0$ and the inequality holds. Suppose, $\alpha \neq \emptyset$, and let $r \neq 0 \in \mathbb{R}$ and we consider

\begin{align*}
  0 \leq (r\alpha + \beta, r\alpha + \beta) & = r^2(\alpha, \alpha) + 2r(\alpha, \beta) + (\beta, \beta) \\
                                            & = a r^2 + 2br + c\text{, where} \\
                                            & a = (\alpha, \alpha), b = (\alpha, \beta), c = (\beta, \beta)
\end{align*}

Well, $ar^2 + 2br + c$ is a quadratic in $r$ that is non-negative.

\begin{align*}
  \rightarrow 4b^2 - 4ac & \leq 0 \\
              b^2 - ac   & \leq 0 \\
              b^2        & \leq ac \\
\\
       (\alpha, \beta)^2 &\leq |\alpha|^2 |\beta|^2
\end{align*}


\subsubsection{Theorem: Triangle Inequality}

For any $\alpha$ and $\beta$ in $V$, $|\alpha + \beta| \leq |\alpha| + |\beta|$

\underline{Proof:}

\begin{align*}
  |\alpha+\beta|^2 = (\alpha+\beta, \alpha+\beta) & = (\alpha, \alpha) + (\beta, \alpha) + (\alpha, \beta) + (\beta, \beta) \\
                                                  & = |\alpha|^2 + 2(\alpha, \beta) + |\beta|^2 \\
                                                  & \leq |\alpha|^2 + 2|\alpha||\beta| + |\beta|^2 \\
                                                  & = (|\alpha| + |\beta|)^2
\end{align*}


\underline{Definition:} Let $V$ be an \textbf{inner product space}, then we say that $\alpha, \beta \in V$ are orthogonal if $(\alpha, \beta) = 0$ 

Example:

In $\mathbb{R}^2$ ($n \leq 3$ by my memory), the ``dot product'' is an inner product. The cross product is NOT an inner product.


\subsubsection{The Gram-Schmidt method}

Recall: We all agreed that a basis is best when it is orthogonal (the basis vectors are of length and all mutually orthogonal).

Q: If we have an arbitrary basis for $V$, can we create from it an orthogonal (even orthogonal) basis? The technique is called \textbf{the Gram-Schmidt method}.

The method:

Suppose $T = {\alpha_1, \alpha_2, ... \alpha_n}$ is a basis for a vector space. We wish to generate a set of vectors ${\beta_1, \beta_2, ... \beta_n}$ that is an orthogonal basis from this basis, T.

\begin{enumerate}
  \item Let $\beta_1 = \alpha_1$. Now, find $\beta_2$ so that $\beta_1 \perp \beta_2$ and $\beta_1, \beta_2$ spans the same subspace as ${\alpha_1, \alpha_2}$ or ${\beta_1, \alpha_2}$. We have: $\beta_2 = a_1\alpha_1 + a_2\alpha_2 = a_1\beta_1 + a_2\alpha_2$

  But we want:
  \begin{align*}
    (\beta_1, \beta_2) = &0 = (\beta_1, a_1\beta_1 + a_2\alpha_2) \\ 
                         &0 = a_1(\beta_1, \beta_1) + a_2(\alpha_2, \beta_1)
  \end{align*}

  Let $a_2 = 1$. Solve for $a_1$.

  \begin{align*}
    a_1 = \frac{-a_2(\alpha_2, \beta_1)}{(\beta_1, \beta_1)}
  \end{align*}

  \begin{align*}
    \beta_2 = \frac{\alpha_2 - a_2(\alpha_2, \beta_1)}{(\beta_1, \beta_1)} \beta_1
  \end{align*}

  \item Next, we want to find $\beta_3$ so that ${\beta_1, \beta_2, \beta_3}$ spans the same subspace as ${\alpha_1, \alpha_2, \alpha_3}$ or ${\beta_1, \beta_2, \alpha_3}$ and they're all orthogonal to each other. We want:

  \begin{align*}
    (\beta_1, \beta_3) &= 0 = b_1(\beta_1, \beta_1) = b_2(\beta_1, \beta_2) + b_3(\beta_1, \alpha_3) \\
    (\beta_2, \beta_3) &= 0 = b_1(\beta_2, \beta_1) = b_2(\beta_2, \beta_2) + b_3(\beta_2, \alpha_3) \\
                       &\text{Let}\ b_3 = 1 \\
                       &b_1 = \frac{-(\beta_1, \alpha_3)}{(\beta_1, \beta_1)}, b_2 = \frac{-(\beta_2, \alpha_3)}{(\beta_2, \beta_1)}
  \end{align*}

  So,

  \begin{align*}
    \therefore \beta_k = \alpha_k - \sum_{h=1}^{k-1} \frac{(\beta_h, \alpha_k)}{(\beta_h, \beta_h)} \beta_h;\ k = 1, ..., n
  \end{align*}

  And, an orthogonal basis is obtained by

  \begin{align*}
    {\frac{\beta_1}{|\beta_1|}, \frac{\beta_2}{|\beta_2|}, ..., \frac{\beta_n}{|\beta_n|}}  
  \end{align*}

  If $S = {\alpha_1, ..., \alpha_n}$ an orthogonal basis for the vector space and $r \in V$ is arbitrary:

  \begin{align*}
    V = a_1\alpha_1 + a_2\alpha_2 + ... + a_n\alpha_n
  \end{align*}

  Finding the multipliers is a cakewalk.

  \begin{align*}
    (\alpha_1, r) &= (\alpha_1, a_1\alpha_1) + (\alpha_1, a_1\alpha_2) + ...
                  &= a_1(a_1\alpha_1) + a_2(a_1\alpha_2)
  \end{align*}

 \end{enumerate} 