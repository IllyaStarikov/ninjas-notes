\section{2018-04-16}

\subsection{The lost assignment - final project}

\subsubsection{}

In most scientific investigations (that are worth a darn) you model reality using partial differential equations (PDEs). So, why? In most studies, you have many variables controlling the system. To model with only one variable is really unrealistic.

A PDE can be characterized as:

An equation relating one or more dependent variables to two or more independent variables and their derivatives of any order with respect to those independent variables. 

Remember how to do partial derivatives of two variables:

\begin{align*}
  z = f(x, y)                                                                                              & = x^2y + y - 2xe^y \\
  \frac{\partial z}{\partial x} = \frac{\partial f}{\partial x}                                            & = 2xy - 2e^y \\
  \frac{\partial f}{\partial y}                                                                            & = x^2 + 1 - 2xe^y \\
  \frac{\partial ^2 f}{\partial y\partial x} = \frac{\partial }{\partial x}(\frac{\partial f}{\partial y}) & = 2x - 2e^y
\end{align*}

Great. Now, here's some equations:

\begin{align*}
  &\frac{\partial^2 u}{\partial x^2} + \frac{\partial ^2 u}{\partial y^2} = f(x, y)    &(x, y) \in \Omega \in \mathbb{R}^2 = \mathbb{R}^\prime  * \mathbb{R}^\prime \\
  &\frac{\partial u}{\partial t} = a \frac{\partial u}{\partial x^2} + f(x, z)          &x \in (0, L) \in \mathbb{R}^\prime ; t > 0 \\
  &\frac{\partial ^2 u}{\partial t^2} = a \frac{\partial ^2 u}{dx^2} + f(x, yz) &x \in (0, L) ; t > 0 
\end{align*}

These are Poisson's equation, the Heat equation, and the Wave equation. In the first case, $f trip 0$, then we call it Laplace's equation. 

What is a solution to a PDE? We're looking for  any function of the required number of independent variables that satisfies the PDE. Solving PDEs is really really... really hard.

We will solve the ``Dirichlet problem'' for Poisson's equation. This means that we know the solution to the PDE on the boundary of the domain.

\begin{align*}
  &u_{xx} + u_{yy} = g(x, y) &\text{for }(x, y) \in \Omega \\
  &\text{with } u(x, y) = f(x, y) \text{ on boundary of } \Omega 
\end{align*}

To make this problem tractable, let's assume that the domain, $\Omega$, is the unit square of $0 \leq x \leq 1$, $0 \leq y \leq 1$. Let's take $N > 0$ to be the number of partitions of both dimensions with $h = 1/N$.

\begin{align*}
  N > 0, N = 4 \\
  \text{So, } (x_j, y_k) = (jh, kh) \\
  0 \leq j, k \leq N
\end{align*}

If we assume that the function we seek is $C^4(\Omega)$ and then we'll see the \underline{$O(h^2)$ centered difference formula} for the approximation to the 2nd derivative of the function, we can use the following derivation for our solution:

\begin{align*}
  G^{\prime\prime}(x) = \frac{G(x+h) - 2 G(x) + G(x-h)}{h^2} - \frac{h^2}{12} G^{(4)}(\eta)
\end{align*}

$\eta$ is necessarily unknown.

If we use this approximation to the 2nd derivative in our PDE in both directions, we got:

\begin{align*}
  \frac{u(x_{j+1}, y_k) - 2 u(x_j, y_k) + u(x_{j-1}, y_k)}{h^2} + \frac{u(x_j, y_{k+1}) - 2 u(x_j, y_k) + u(x_j, y_{k-1})}{h^2} \\
  = g(x_j, y_k) - \frac{h^2}{12} (\frac{\partial^4 u}{\partial y^4}(\zeta, \eta))
\end{align*}

Let $u_h(x_j, y_k) = f(x, y)$ on the boundary of $\Omega$. Dropping the approximating term on the right-hand side, we have an approximation for the solution, $u(x, y)$ on the interior of $\Omega$.

After simplification:

\begin{align*}
  u_h(x_j, y_k) = \frac{1}{4} ( u_h(x_{j+1}, y_k) + u_h(x_{j-1}{y_k}) + u_h(x_j, y_{k-1}) + u_h(x_j, y_{k+1}) ) - \frac{h^2}{4} g(x_j, y_k)
\end{align*}

This equation gives us $(N-1)^2$ equations in $(N-1)^2$ unknowns.